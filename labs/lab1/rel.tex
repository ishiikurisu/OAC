\documentclass[12pt, a4paper, twoside]{article}
\usepackage[utf8]{inputenc}
\usepackage[cm]{fullpage}
\usepackage{fancyhdr}
\usepackage{textcomp}
\usepackage{graphicx}

\begin{document}

\title{Relatório do Experimento 1 de OAC}
\author{
Cristiano Silva Júnior: 13/0070629 \\
Cristiano Silva Júnior: 13/0070629 \\
Cristiano Silva Júnior: 13/0070629 \\
Cristiano Silva Júnior: 13/0070629}
\date{5 de Maio de 2017}
\maketitle

\section{Exercício 1}

\subsection{Exercício 1.1}

Lendo o programa \textit{sort.s} dado, nota-se que ele vai realizar um  ordenamento decrescente devido à única comparação presente no código estar  invertida. Após inverter os registradores para realizar a comparação correta, podemos analisar o programa usando a ferramenta \textit{Instruction Counter} do \textit{Mars}, contamos 551 instruções, sendo 204 do tipo R, 294 do tipo I e 53 do tipo J para o vetor dado. Utilizando a ferramenta de estatísticas \textit{Instructions statistics}, foram 31\% de instruções de ULA; 13\% do tipo \textit{jump}; 13\% do tipo\textit{branch}; 27\% de memória; e 16\% de outros tipos.

\subsection{Exercício 1.2}

Reutilizando o programa \textit{sort.s}, podemos analisar este algoritmo de ordenamento para outras entradas. Considerando as entradas $v_o(n) = {1,2,3,...,n}$ e $v_i(n)={n,n-1,n-2,...,1}$ para $n={1,2,3,4,5,6,7,8,9,10,20,30,40,50,60,70,80,90,100}$, podemos calcular o número de instruções para cada $n$ e calcular o tempo de execução $t$ com os valores sugeridos de frequência de clock e de CPI. A relação $n \cdot t$ na figura 1.

% TODO Calcular n x t

\section{Exercício 2}

Falar sobre os programas compilados.

\section{Exercício 3}

Falar sobre como desenhamos os pontos, as linhas e tudo mais.

\end{document}
