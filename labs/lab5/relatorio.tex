\documentclass[12pt, a4paper, twoside]{article}
\usepackage[utf8]{inputenc}
\usepackage[cm]{fullpage}
\usepackage{fancyhdr}
\usepackage{textcomp}
\usepackage{graphicx}

\begin{document}

\title{Relatório do Experimento 3 de OAC}
\author{
Arthur Bizzi: 13/0102636 \\
Arthur da Silveira Couto: 16/0002575 \\
Caio Albuquerque Brandão: 16/0003636 \\
Cristiano Silva Júnior: 13/0070629 \\
Leonardo Maffei: 16/0033811 \\}
\date{7 de Junho de 2017}
\maketitle

\section{Exercício 1}

% TODO Testar programa testePIPE.s

\section{Exercício 2}

% TODO Diagrama de blocos do caminho de dados
% TODO Tabela verdade da unidade de controle

\section{Exercício 3}

% TODO Analisar unidades de Hazard e Forward

\section{Exercício 4}

% TODO Simular forma de onda do programa teste.s

\section{Exercício 5}

% TODO Linkar vídeos de desenho das bandeiras

\section{Exercício 6}

% TODO Incluir bolhas para realizar operações de ponto flutuante

\section{Exercício 7}

% TODO Comparar síntese da FPULA

\section{Exercício 8}

% TODO Implementar ceil
% TODO Implementar floor
% TODO Implementar round

\end{document}
