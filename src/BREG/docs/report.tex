\documentclass[12pt, a4paper, twoside]{article}
\usepackage[utf8]{inputenc}
\usepackage[cm]{fullpage}
\usepackage{fancyhdr}
\usepackage{textcomp}
\usepackage{graphicx}
\usepackage{commath}
\usepackage[portuguese]{babel}

\begin{document}

\title{Trabalho 5 da disciplina "Organização e Arquitetura de Computadores" 1 /
2018}
\author{Cristiano Silva Júnior: 13/0070629}
\date{\today}
\maketitle

\section{Introdução}

O objetivo deste trabalho é implementar o banco de registradores (BREG) do
processador MIPS desenvolvido na disciplina. O BREG deverá permitir a
recuperação de dados salvos durante a execução de um problema. Os dados somente
poderão ser salvos quando um indicador \textit{"write enable"} for verdadeiro.
Além disso, deve haver uma opção para apagar todos os dados no registrador.

\section{Metodologia}

A implementação do BREG foi feita em VHDL utilizando a ferramenta Altera Quartus
II com o auxílio do Altera ModelSim para a suíte de testes. Testes unitários
para cada operação estão contidos no arquivo "testbench.vhd" enquanto o BREG
foi implementado no arquivo "BREG.vhd".

\section{Resultados}

\subsection{Circuito gerado}

Por meio da implementação, pode-se gerar o circuito indicado na figura 1,
que contém X componentes.

% TODO Gerar vizualização do circuito.

\subsection{Simulação}

Utilizando o ModelSim, geramos uma simulação em forma de onda para o circuito.
No caso, mostramos a dinâmica recomendada de atribuir um valor arbitrário aos
registradores que depois são lidos em saídas diferentes

% TODO Gerar visualização da forma de onda do circuito.

\section{Conclusão}

A implementação de todao BREG proposta foi possível sem maiores problemas.

\end{document}
