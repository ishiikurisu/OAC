\documentclass[12pt, a4paper, twoside]{article}
\usepackage[utf8]{inputenc}
\usepackage[cm]{fullpage}
\usepackage{fancyhdr}
\usepackage{textcomp}
\usepackage{graphicx}
\usepackage{commath}
\usepackage[portuguese]{babel}
\usepackage{amssymb}
\usepackage{listings}

\begin{document}

\title{Trabalho 2 de Organização e Arquitetura de Computadores 1/2018}
\author{Cristiano Silva Júnior: 13/0070629}
\date{\today}
\maketitle

\section{Introdução}

Este relatório explica como foi implementado um simulador de programas em
linguagem de máquina \textit{MIPS}. O objetivo deste projeto foi fazer um
programa que fosse capaz de executar um mínimo de instruções que bastasse
para alguns casos de teste, já que a implementação de um simulador completo é
muito grande para o escopo da disciplina.

Em termos mais abstratos, um processador \textit{MIPS} funciona executando um
\textit{loop} básico, onde a máquina carrega instruções e dados de suas
memórias correspondentes; decodifica as instruções em variáveis; e as executa,
podendo ou não salvar dados em seu banco de registradores ou na própria
memória.

O carregamento de instruções é baseado em um registrador especial do
processador chamado $PC$, que é responsável por guardar a posição atual do
programa em memória. Após cada carregamento, este registrador é incrementado
em um valor constante que indica a próxima instrução.

Na linguagem MIPS, as instruções podem ser dividas em três tipos:

\begin{itemize}
    \item Tipo R, que contém 6 campos codificados (\textit{opcode, rs, rt,
    rd, shamt e funct}) e é destinado para operações em geral, principalmente
    as de matemática.
    \item Tipo I, que contém 4 campos (\textit{opcode, rs, rt e imediato}) e
    é destinado para operações em geral que contém uma constante como
    argumento.
    \item Tipo J, que contém 2 campos (\textit{opcode e endereço}) e é
    destinado para operações voltadas para alterar o fluxo do programa.
\end{itemize}

Os campos codificados são extraídos na etapa de decodificação, incluíndo o
\textit{opcode}, que identifica a instrução a ser executada na etapa seguinte.
Os campos \textit{rs, rt e rd} identificam os registradores a serem operados;
enquanto os campos \textit{imediato e endereço}, que são constantes relevantes
ao contexto do programa.

\section{Metodologia}

Para melhor simular um processador MIPS, desenvolvemos um programa em linguagem
\textit{GNU C} compilados com \textit{GCC} e \textit{Make} tanto em Microsoft
Windows 8 e Arch Linux. O foco principal neste programa encontra-se em uma
entitade doravante denominada \textit{processador}, cujo objetivo é executar
três procedimentos:

\begin{itemize}
    \item Carregar a próxima instrução em memória baseado no processador $PC$;
    \item Decodificar a instrução, carregando os parâmetros $RT$, $RS$, $RD$,
    $opcode$, $funct$, $shamt$ e $imm$ na memória;
    \item Executar a instrução atual baseado nos parâmetros definidos naquele
    momento.
\end{itemize}

A implementação do processador encontra-se em 2 arquivos separados:

\begin{itemize}
    \item Uma biblioteca \textit{mips.h} que contém diversas funções cujos
    objetivos são auxiliar o processador em tarefas relacionadas à linguagem
    \textit{MIPS};
    \item O simulador em si, que consiste de uma \textit{struct} do C com
    banco de registradores, memória de instruções, memória de dados e algumas
    variáveis relevantes para a execução do programa. Com excessão de um
    interruptor para indicar o fim da execução de um programa, todos as
    propriedades foram implementadas utilizando inteiros de 32 bits como
    implementados na biblioteca padrão \textit{stdint.h}.
\end{itemize}

O processador deverá executar arquivos binários de programas \textit{MIPS}
compilados na plataforma \textit{MARS Simulator v4.5} como indicados em
requerimentos descritos no arquivo \textit{makefile}.

Com isto em mente, o programa pode ser compilado e executado por meio da
seguinte chamada no diretório raíz:
\begin{lstlisting}[caption=Compilação do simulador MIPS]
$ make
\end{lstlisting}
Ela gerará os binários MIPS relevantes; executará uma rotina de testes
unitários nas bibliotecas do processador; compilará o programa do processador;
e o executará.
Certifique-se que a plataforma $MARS$ está acessível pela linha de comando
para que a compilação funcione.

\section{Resultados}

A fim de validar a aplicação desenvolvida, o programa foi testado em dois
programas de exemplo em linguagem \textit{MIPS}.

\subsection{Programa de Exemplo 1}

\begin{lstlisting}[caption=Primeiro teste]
.data
CONST: .word 5

.text
# read a
la $t0, CONST
lw $t0, 0($t0)

# read b
li $v0, 5
syscall
add $t1, $v0, $zero

# sum a + b
add $a0, $t1, $t0
li $v0, 1
syscall

# exit
li $v0, 10
syscall
\end{lstlisting}

Este programa deverá ler um número inteiro do teclado e apresentar a sua soma
com 5. Os testes em ambos sistemas operacionais funcionaram.

\subsection{Programa de Exemplo 2}

O segundo programa a ser testado foi um teste enviado pelo professor como
exemplo e já contém algumas operações mais complexas.

\begin{lstlisting}[caption=Segundo teste]
.data
primos: .word 1, 3, 5, 7, 11, 13, 17, 19
size: .word 8
msg: .asciiz "Os oito primeiros numeros primos sao : "
space: .ascii " "
.text
 la $t0, primos #carrega endereco inicial do array
 la $t1, size #carrega endereco de size
 lw $t1, 0($t1) #carrega size em t1
 li $v0, 4 #imprime mensagem inicial
 la $a0, msg
 syscall
loop: beq $t1, $zero, exit #se processou todo o array, encerra
 li $v0, 1 #servico de impressao de inteiros
 lw $a0, 0($t0) #inteiro a ser exibido
 syscall
 li $v0, 4 #imprime separador
 la $a0, space
 syscall
 addi $t0, $t0, 4 #incrementa indice array
 addi $t1, $t1, -1 #decrementa contador
 j loop #novo loop
exit: li $v0, 10
 syscall
\end{lstlisting}

Este programa deverá escrever na saída padrão o seguinte texto:

\begin{lstlisting}
os oitos primeiros numeros primos sao : 1 3 5 7 11 13 17 19
\end{lstlisting}

% BUG Fix this!
Em testes em ambos os sistemas operacionais, não foi possível escrever os
caracteres codificados em números na memória \textit{MIPS} por erros não
compreendidos.

\section{Conclusão}

Com a implementação do simulador de um processador \textit{MIPS}, podemos
aprender como funciona o procedimento básico de execução da arquitetura, que
consiste em 3 etapas simples. Estas etapas requerem que as instruções sejam
executadas com manipulação de registradores e acesso à memória.

\section{Referência Bibliográfica}

\begin{enumerate}
    \item "MIPS32® Architecture For Programmers". Volume II.
\end{enumerate}

\end{document}
