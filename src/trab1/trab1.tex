\documentclass[12pt, a4paper, twoside]{article}
\usepackage[utf8]{inputenc}
\usepackage[cm]{fullpage}
\usepackage{fancyhdr}
\usepackage{textcomp}
\usepackage{graphicx}
\usepackage{commath}
\usepackage[portuguese]{babel}
\usepackage{amssymb}

\begin{document}

\title{Trabalho 1 de Organização e Arquitetura de Computadores 1/2018}
\author{Cristiano Silva Júnior: 13/0070629}
\date{\today}
\maketitle

\section{Introdução}

Este é o relato do projeto 1, onde foi pedido para implementar algumas funções
de acesso à memória em C utilizando a mesma lógica de acesso da ISA MIPS. A
saber, foram determinadas as seguintes funções para implementação:

\begin{itemize}
    \item $lw$, que lê uma \textit{word} (4 \textit{bytes}) da memória;
    \item $lh$, que lê uma \textit{half word} (2 \textit{bytes}) com sinal;
    \item $lhu$, que lê uma \textit{half word} sem sinal;
    \item $lb$, que lê 1 \textit{byte};
    \item $lbu$, que lê 1 \textit{byte} sem sinal;
    \item $sw$, que escreve uma \textit{word} na memória;
    \item $sh$, que escreve uma \textit{half word} na memória;
    \item $sb$, que escreve um \textit{byte} na memória;
\end{itemize}

Os parâmetros das funções em C devem imitar os parâmetros das funções na
linguagem \textit{assembly} original.

Aqui, cabe esclarecer que um \textit{byte} será definido, para os fins
deste projeto, como uma lista de tamanho 8 de variáveis booleanas
(indivualmente chamadas de bits). Logo, uma \textit{word}, que contém 4
\textit{bytes}, é uma lista de 32 bits. Considerando um \textit{byte} $b$ como
sendo $$ b := \{ b_i \}^{8}_{i=1} $$, então algumas operações comumente
definidas para variáveis booleanas podem ser extrapoladas para bytes. No caso,
vamos trabalhar com as operações "e" e "ou", definidas respectivamente como:
$$ a \land b := \{ a_i \land b_i \}^{8}_{i=1} $$
$$ a \lor b := \{ a_i \lor b_i \}^{8}_{i=1} $$
Uma nova operação a ser definida é o deslocamento (\textit{shift}), que pode
ser escrito como sendo direcionado para a direita:
$$ b \gg x := \{ \top \land b_{i-x} \}^{8}_{i=1} $$
ou para a esquerda:
$$ b \ll x := \{ \top \land b_{i+x} \}^{8}_{i=1} $$
onde $ x \in \mathbb{N} $.

Neste contexto, vamos definir que -1 representa a lista onde todos os elementos
são $ \top $, ou seja, todos os bits são verdadeiros. Além disso, 0 representa

\section{Metodologia}

O principal objeto a ser manipulado neste procedimento é a memória. No caso,
resolvi implementar a memória como um \textit{array} de \textit{words} com um
tamanho fixo de 4096 \textit{words}. As instruções na ISA MIPS acessam a
memória \textit{byte} a \textit{byte} [1], ou seja, cada endereço se refere a
um byte diferente. Desta forma, para ler e escrever na memória, certas
operações se fazem necessárias para implementar as funções.

Para ler \textit{bytes} e da memória, preferi utilizar uma lógica de máscara.
O valor lido $v$ do $k$-ésimo byte de um endereço $a$ da memória de words $M$
será
$$ v = \left( M \lfloor a \rfloor \gg 8k \right) \land -1 $$
indicando que o valor lido originalmente da memória será ajustado e mascarado
para se recuperar o \textit{byte} em questão. A mesma operação pode ser
utilizada para \textit{half words}.

Para escrever na memória, a mesma operação de leitura pode ser utilizada, mas
agora adaptando a máscara para o deslocamento de endereço desejado e levando o
dado a ser guardado em consideração. Neste caso, a função do valor guardado $v$
para um dado $d$ será:
$$ v = \left( M[a] \land P \right) \lor d \ll 8k $$
onde $P$ é a máscara correspondente para o deslocamento necessário. Essa
operação também pode ser utilizada para \textit{half words} porém máscaras
próprias deverão ser utilizadas.

Essas funções assumem algumas situações:

\begin{itemize}
    \item O endereço $a$ se refere a uma posição de um byte dentro da memória;
    \item $k$ é positivo, e menor que 4 para \textit{bytes} ou par para
    \textit{half words}. Valores fora dessa faixa sempre retornam o valor 0.
\end{itemize}

\section{Resultados}

A implementação encontra-se no código fonte em anexo ao trabalho. O código é
escrito em \textbf{ANSI C} e pode ser compilado chamando o programa $make$ em
qualquer sistema operacional com o pacote \textbf{GCC}.

\section{Conclusão}

As operações realizadas no trabalho serviram para modelar matematicamente o
acesso à memória realizada por operações da ISA MIPS. Nota-se que o
endereçamento original, apesar de ser feito a \textit{word}, precisa respeitar
a posição dos \textit{bytes} individuais, necessitando de atenção na construção
do hardware necessário.

\section{Referência Bibliográfica}

\begin{enumerate}
    \item "MIPS32® Architecture For Programmers". Volume II.
\end{enumerate}

\end{document}
