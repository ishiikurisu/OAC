\documentclass[12pt, a4paper, twoside]{article}
\usepackage[utf8]{inputenc}
\usepackage[cm]{fullpage}
\usepackage{fancyhdr}
\usepackage{textcomp}
\usepackage{graphicx}
\usepackage{commath}
\usepackage[portuguese]{babel}

\begin{document}

\title{Trabalho 4 da disciplina "Organização e Arquitetura de Computadores" 1 /
2018}
\author{Cristiano Silva Júnior: 13/0070629}
\date{\today}
\maketitle

\section{Introdução}

O objetivo deste trabalho é implementar a unidade lógico-aritmética (ULA) do
processador MIPS desenvolvido na disciplina. A ULA deverá implementar as
seguintes operações:

\begin{itemize}
    \item Soma e subtração aritméticas
    \item Soma e subtração lógicas;
    \item AND, OR, NOR e XOR lógicos;
    \item "Menor que";
    \item \textit{Shift} e rotação lógicos para esquerda e direita;
\end{itemize}

A ULA deverá receber como entrada dois números $A$ e $B$ de 32 bits; e um
código para oeração definido previamente pelo professor e deverá ter como saída
um número $Z$ com o resultado da operação escolhida; um indicador se o
resultado vale zero; e outro indicador para a situação de overflow na operação.

\section{Metodologia}

A implementação da ULA foi feita em VHDL utilizando a ferramenta Altera Quartus
II com o auxílio do Altera ModelSim para a suíte de testes. Testes unitários
para cada operação estão contidos no arquivo "testbench.vhd" enquanto a ULA
foi implementada no arquivo "ALU.vhd".

\section{Resultados}

\subsection{Circuito gerado}

Por meio da implementação, pode-se gerar o circuito indicado na figura 1.

% TODO Gerar vizualização do circuito.

\subsection{Operações aritméticas}

Todas as operações aritméticas foram testadas para resultados incluindo
números positivos, negativos e zeros, assim como para mostrar a detecção de
overflow como mostra a figura 2.

% TODO Mostrar simulação para operações aritméticas.

\subsection{Operações lógicas}

Todas as operações lógicas foram testadas com números com os quais seriam
fáceis de visualizar o resultado final, como mostra a figura 3.

% TODO Mostrar simulação para operações lógicas.

Além disso, implementamos as operações de soma e subtração para os casos
com números sem sinal, como mostra o resultado da simulação na figura 4.

% TODO Mostrar simulação para soma e subtração lógicas.

\subsubsection{Operações "Menor que"}

Para as operações "menor que", foram utilizas duas situações: uma em que
o número $A$ é de fato menor que $B$, tornando $Z=1$; e outra em que
$A>B \rightarrow Z=0$. O resultado da simulação encontra-se na figura 5.

% TODO Mostrar simulação para operações <

\subsection{Operações de \textit{Shift} e rotações}

As operações de \textit{shift} e de rotação foram testadas com casos
extremos, isto é, que há interações do resultado com o limite da representação
númerica de 32 bits, como mostra a figura 6.

% TODO Mostrar simulação para operações de shift e rotação

\section{Conclusão}

A implementação de toda a ULA proposta foi possível sem maiores problemas.s

\end{document}
