\documentclass[12pt, a4paper, twoside]{article}
\usepackage[utf8]{inputenc}
\usepackage[cm]{fullpage}
\usepackage{fancyhdr}
\usepackage{textcomp}
\usepackage{graphicx}
\usepackage{commath}
\usepackage[portuguese]{babel}

\begin{document}

\title{Trabalho 3 da disciplina "Organização e Arquitetura de Computadores" 1 /
2018}
\author{Cristiano Silva Júnior: 13/0070629}
\date{\today}
\maketitle

\section{Introdução}

O objetivo deste trabalho é implementar algumas operações de aritmética de
números em ponto flutuante na linguagem MIPS sem utilizar as facilidades
fornecidas pelo coprocessador 1, especializado em números deste tipo. No caso,
devemos implementar as operações de soma; subtração; e multiplicação utilizando
somente operações comuns do MIPS MARS.

A saber, um número em ponto flutuante é uma representação numérica padronizada
no documento IEEE 754, e consiste de um número de 32 bits dividido como
mostrado na figura 1.

% TODO Add IEEE 754 float bit division

No caso, o número representado $N$ será [1] $$ N = (-1)^S (1 + C) \cdot 2^{M-127} $$

Como os números em ponto flutuante na verdade representam números em notação
científica, as operações de soma e multiplicação devem, de fato, implementar os
respectivos algoritmos para números neste formato, como demonstrado na figura 2.

% TODO Add sum and multiplication procedures

A subtração será similar a soma, já que basta inverter o sinal do número
seguindo a representação em complemento de dois [2].

\section{Metodologia}

A partir desta seção, denominaremos números em ponto flutuante de precisão
simples como \textit{floats} para fins de brevidade.

Os procedimentos de soma e de multiplicação descritos anteriormente possuem
algumas operações em comum, como extração dos expoentes; das mantissas;
normalização de um número; extração do sinal; e transformação em complemento
de 2. Neste caso, decidimos implementar essas operações primeiro antes de
partir para um programa principal, cujo objetivo é ler dois números em ponto
flutuante; a operação desejada; e escrever na saída padrão o resultado
prometido.

A extração do sinal de um \textit{float} é simples: basta descobrir o bit mais
significativo do número. Isto é possível realização um \textit{shift} lógico à
direita com 31 passos.

A extração da mantissa de um \textit{float} é igualmente simples e pode ser
feita realizando uma operação $AND$ entre número em questão e uma máscara de 22
bits. Em hexadecimal, essa máscara será $7FFFFF$.

A extração do expoente é um pouco mais complicada e requer tanto um
\textit{shift} lógico à direita (no caso, de 23 passos) e um mascaramento (no
caso, com uma máscara de 8 bits, ou $FF$ em hexadecimal).

% TODO Implement 2-complement transformation

% TODO Implement normalization procedure

A operação de normalização de um número em ponto flutuante demonstrou ser,
entre os procedimentos auxiliares, o mais difícil devido à diversidade de casos
de usos diferentes a serem tratados.

\section{Resultados}

Dois programas demonstrativos em MIPS foram escritos utilizando a plataforma
MARS. Um deles descreve testes unitários para as sub-operações necessárias
para se realizar os procedimentos foco deste trabalho e está escrito no arquivo
\textit{test.asm}. O outro é o programa principal, que busca cumprir com a
promessa de ler dois números e realizar uma operação sobre eles, escrito no
arquivo \textit{main.asm}.

Os testes unitários mostram como as operaçãoes de extração de sinal; de
mantissa; e de expoente são diretas e curtas funcionando bem para o caso de
exemplo $-3,14159$ (ou $C0490FD0$ em hexadecimal).

Como foi entregue, o programa principal consegue ler dois números; um símbolo
para a operação desejada; e chamar o procedimento desejado. Contudo, não foi
possível realizar as operações ainda porque eu não sei normalizar um número.

% TODO Compare results with default functions.

\section{Conclusão}

A partir do entendimento dos algoritmos de soma e multiplicação de números em
notação científica, foi possível implementar em linguagem MIPS procedimentos
capazes de substituir as funções originais. Apesar de existir um erro entre os
resultados das implementações, vimos que ele é relativamente pequeno e não
impede o seu uso.

\section{Referência Bibliográfica}

\begin{enumerate}
    \item "754-2008 - IEEE Standard for Floating-Point Arithmetic".
    https://ieeexplore.ieee.org/document/4610935/. Último acesso em 4 de Maio
    de 2018.
    \item Adicionar referência a um número em complemento de 2 % TODO Add reference
\end{enumerate}

\end{document}
